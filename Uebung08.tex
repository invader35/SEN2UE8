\documentclass[a4paper,oneside,openany]{tufte-book}
\usepackage[utf8x]{inputenc}
\usepackage[german]{babel}
\usepackage{microtype} % Improves character and word spacing
\usepackage{booktabs} % Better horizontal rules in tables
\usepackage{ucs}
\usepackage{amsmath}
\usepackage{mathtools}
\usepackage{amsfonts}
\usepackage{amssymb}
\usepackage{graphicx}
\usepackage{color}
\usepackage{listings}
\usepackage{caption}
\usepackage{upgreek}

\makeatletter% since we're using commands with @ in their name

\let\origappendix\appendix% save a copy of the original meaning of \appendix
\renewcommand{\appendix}{%
  \origappendix% do all the original \appendix stuff
  \titlecontents{chapter}%
    [0em] % distance from left margin
    {\vspace{1.5\baselineskip}\begin{fullwidth}\LARGE\rmfamily\itshape} % above (global formatting of entry)
    {\hspace*{0em}\appendixname~\thecontentslabel: } % before w/label (label = ``2'')
    {\hspace*{0em}} % before w/o label
    {\rmfamily\upshape\qquad\thecontentspage} % filler + page (leaders and page num)
    [\end{fullwidth}] % after
  \titleformat{\chapter}%
    [display]% shape
    {\relax\ifthenelse{\NOT\boolean{@tufte@symmetric}}{\begin{fullwidth}}{}}% format applied to label+text
    {\itshape\huge Anhang~\thechapter}% label
    {0pt}% horizontal separation between label and title body
    {\huge\rmfamily\itshape}% before the title body
    [\ifthenelse{\NOT\boolean{@tufte@symmetric}}{\end{fullwidth}}{}]% after the title body
  \setcounter{secnumdepth}{0}% ``number'' the appendices
  \renewcommand{\thefigure}{\@arabic\c@figure}% define \thefigure to use only the figure number (1), not A.1
  \renewcommand{\thetable}{\@arabic\c@table}%
  %
  % Add any other special appendix-related code here.
  %
}
\makeatother% restore the special meaning of @


\author{Schett Matthias}
\title{SEN-\"{U}bung 08}

\begin{document}

\definecolor{gray}{rgb}{0.9,0.9,0.9} % background color for the listings

\lstset{language=[Visual]Basic, morekeywords={param, local}, breaklines=true, tabsize=4, showstringspaces=false,backgroundcolor=\color{gray}, numbers=left,
    basicstyle=\ttfamily} % standard listing settings

\frontmatter

\maketitle
\tableofcontents
\mainmatter

\chapter{Aufgabe 1}

\section{L\"{o}sungsidee}

Der Shift Algorithmus erstellt einen Iterator der auf die count .te Stelle im Container zeigt und rotiert dann den Container mit std::rotate, anschließend werden die count letzten Stellen mit der Wert von val
aufgefüllt.
Der Iterator für die Mitte und die count letzte Stelle wird mittels std::advance erstellt.
Der Code findet sich ab Anhang \nameref{chap:Auf1}.

\section{Testf\"{a}lle}
\begin{fullwidth}
\lstinputlisting[caption=Ausgabe des Testreibers]{TemplateShift/OutputA1.txt}
\end{fullwidth}
\chapter{Aufgabe 2}

\section{L\"{o}sungsidee}

Die Funktion checkSorting prüft die Sortierung in einem per Iteratoren gegebenen Bereich und gibt anschließend die vier in der Aufgabenstellung definierten Zustände zurück.
Erreicht wird die Prüfung durch eine Iteration über den gegebenen Bereich und der Prüfung ob der Iterator größer oder kleiner ist als der nachfolgende, entsprechend werden Zähler erhöht.
Nach der Iteration wird, falls ein Zähler gleich groß wie der gegebene Bereich ist, das Ergebnis Ascending, Descending oder EqualOrEmpty zurückgegebenen. Sollte kein Zähler das Kriterium erfüllen,
dann kann davon ausgegangen werden, dass der Bereich nicht sortiert wird und daher wird UnSorted zurückgegeben.
Es gibt auch eine Version mit Prädikat, bei dieser verhält es sich gleich nur statt dem Vergleich wird das Ergebnis des Prädikates, welches als bool vorliegen muss, verwendet.
Der Code findet sich ab Anhang \nameref{chap:Auf2}.

\section{Testf\"{a}lle}
\begin{fullwidth}
\lstinputlisting[caption=Ausgabe des Testreibers]{GenericSortCheck/OutputA2.txt}
\end{fullwidth}
\chapter{Aufgabe 3}

\section{L\"{o}sungsidee}

Allgemein lässt sich die Laufzeikomplexität nach folgender Formel bestimmen:

\[
T(n) = a * T(\frac{n}{b}) + f(n)
\]

Wobei \textbf{a} für die Anzahl and rekursiven Aufrufen steht, $\frac{n}{b}$ für die Anzahl an Unterproblemen und
\textbf{f(n)} für den Aufwand der Aufteilung des Problemes und zur Anschließenden Kombination der Teillösungen.
Es gibt 3 Lösungsvarianten für diese Gleichung:

\begin{tabular}{l | p{5cm} | l}
Fall & Wenn... & Dann.. \\
\hline
1 & $ f(n) = \mathcal{O}(n ^{\log_b{(a-\epsilon})}) $ für $ \epsilon > 0 $ & $ T(n) = \Uptheta(n^{\log_b{(a)}}) $\\
\hline
2 & $ f(n) = \Uptheta(n^{\log_b{(a)}}) $ & $ T(n) = \Uptheta(n^{\log_b{(a)}} * \log{(n)}) $ \\
\hline
3 & $ f(n) = \Omega(n ^{\log_b{(a-\epsilon})}) $ für $ \epsilon > 0 $ % 3.Zeile
und \newline
$ a*f(\frac{n}{b}) \leq c*f(n) $ für $ c < 1 $
& $ T(n) = \Uptheta(f(n)) $ \\ % 3.Zeile
\end{tabular}

Es wird nun für das erste und das zweite Beispiel der Lösungsweg skiziert für die restlichen Beispiele wird anschließend nur die Lösung angegeben.

\subsection{Beispiel 1}

Angabe: $ T(n) = 3T(\frac{n}{2}) + n^2 $

Daher ist $a = 3, b=2, f(n) = n^2 $ und der $ \log_b{(a)} \approx 1.6 $

$ n^2 \in \Upomega(n^{1.6+\epsilon})$

und $ 3 * (\frac{n}{2})^2 \leq cn^2 $ berrechnet man nun den ersten Ausdruck erhält man

$ \frac{3n^2}{4} \leq cn^2 $ für c = 0.9 z.Bsp

Daher ist die Lösung vom Fall 3 $ T(n) = \Uptheta(n^2) $

\subsection{Beispiel 2}

Angabe: $ T(n) = 12T(\frac{n}{4}) + n $

Daher ist $a = 12, b=4, f(n) = n $ und der $ \log_b{(a)} \approx 1.8 $

$ n \in \mathcal{O}(n^{1.8-\epsilon})$

Daher ist die Lösung vom Fall 1 und $ T(n) = \Uptheta(n^{1.8}) $
\newpage
\subsection{Restliche Beispiele}

Hier folgen nun die Lösungen der restlichen Beispiele:

\begin{enumerate}
\setcounter{enumi}{2}
  \item $ T(n) = \Uptheta(n) $ \hspace*{2em} Fall 3
  \item $ T(n) = \Uptheta(n^{1.5}) $ \hspace*{2em} Fall 1
  \item $ T(n) = \Uptheta(n^2) $ \hspace*{2em} Fall 3
  \item $ T(n) = \Uptheta(n^3 * \log{(n)}) $ \hspace*{2em} Fall 2
\end{enumerate}

\backmatter

\appendix

\setboolean{@mainmatter}{true}
\chapter{Aufgabe 1}\label{chap:Auf1}

\begin{fullwidth}
\lstinputlisting[caption=Implementierung der Shiftfunktion]{TemplateShift/TemplateShift.h}
\lstinputlisting[caption=Testtreiber]{TemplateShift/Main.cpp}
\end{fullwidth}

\chapter{Aufgabe 2}\label{chap:Auf2}

\begin{fullwidth}
\lstinputlisting[caption=Implementierung der Sortierprüfung]{GenericSortCheck/GenericSortCheck.h}
\lstinputlisting[caption=Testtreiber]{GenericSortCheck/Main.cpp}
\end{fullwidth}

\end{document}